\begin{figure}[t]

\begin{minipage}[t]{\linewidth}

{\centering 

\begin{Shaded}
\begin{Highlighting}[]
\ImportTok{from}\NormalTok{ poincare }\ImportTok{import}\NormalTok{ Derivative, System, Variable, initial}
\end{Highlighting}
\end{Shaded}

}

\end{minipage}%
\newline
\begin{minipage}[t]{\linewidth}

{\centering

\begin{minipage}[c]{0.6\linewidth}

{\centering 

\begin{Shaded}
\begin{Highlighting}[]
\KeywordTok{class}\NormalTok{ Decay(System):}
\NormalTok{    x: Variable }\OperatorTok{=}\NormalTok{ initial(default}\OperatorTok{=}\DecValTok{1}\NormalTok{)}
\NormalTok{    eq }\OperatorTok{=}\NormalTok{ x.derive() }\OperatorTok{\textless{}\textless{}} \OperatorTok{{-}}\NormalTok{x}
\end{Highlighting}
\end{Shaded}

}

\end{minipage}%
%
\begin{minipage}[c]{0.2\linewidth}

{\centering 

\[
\frac{dx}{dt} = -x
\]

}

\end{minipage}%
%
\begin{minipage}[c]{0.2\linewidth}

{\centering 

\[
x(0) = 1
\]

}

\end{minipage}%

\subcaption{First-order system of an exponential decay. The variable
\texttt{x} stores the initial condition for \(x\), and the variable
\texttt{eq} stores the rate equation for \(x\).}

}

\end{minipage}%
\newline
\begin{minipage}[t]{\linewidth}

{\centering 

\begin{minipage}[c]{0.6\linewidth}

{\centering 

\begin{Shaded}
\begin{Highlighting}[]
\KeywordTok{class}\NormalTok{ Oscillator(System):}
\NormalTok{    x: Variable }\OperatorTok{=}\NormalTok{ initial(default}\OperatorTok{=}\DecValTok{1}\NormalTok{)}
\NormalTok{    v: Derivative }\OperatorTok{=}\NormalTok{ x.derive(initial}\OperatorTok{=}\DecValTok{0}\NormalTok{)}
\NormalTok{    eq }\OperatorTok{=}\NormalTok{ v.derive() }\OperatorTok{\textless{}\textless{}} \OperatorTok{{-}}\NormalTok{x}
\end{Highlighting}
\end{Shaded}

}

\end{minipage}%
%
\begin{minipage}[c]{0.2\linewidth}

{\centering 

\[
\frac{d^2x}{dt^2} = -x
\]

}

\end{minipage}%
%
\begin{minipage}[c]{0.2\linewidth}

{\centering 

\[
\begin{cases}
    x(0) &= 1 \\
    \frac{dx}{dt}(0) &= 0
\end{cases}
\]

}

\end{minipage}%

\subcaption{Second-order system of an harmonic oscillator, where the
variable \texttt{v} stores the derivative \(\frac{dx}{dt}\) and the rate
equation is specified for the derivative \texttt{v}.}

}

\end{minipage}%
\newline
\begin{minipage}[t]{\linewidth}

{\centering 

\begin{minipage}[c]{0.6\linewidth}

{\centering 

\begin{Shaded}
\begin{Highlighting}[]
\KeywordTok{class}\NormalTok{ BigModel(System):}
\NormalTok{    x: Variable }\OperatorTok{=}\NormalTok{ initial(default}\OperatorTok{=}\DecValTok{1}\NormalTok{)}
\NormalTok{    linked }\OperatorTok{=}\NormalTok{ Decay(x}\OperatorTok{=}\NormalTok{x)}
\NormalTok{    independent }\OperatorTok{=}\NormalTok{ Decay(x}\OperatorTok{=}\DecValTok{2}\NormalTok{)}
\end{Highlighting}
\end{Shaded}

}

\end{minipage}%
%
\begin{minipage}[c]{0.2\linewidth}

{\centering 

\[
\begin{cases}
    \frac{dx}{dt} = -x \\
    \frac{dy}{dt} = -y
\end{cases}
\]

}

\end{minipage}%
%
\begin{minipage}[c]{0.2\linewidth}

{\centering 

\[
\begin{cases}
    x(0) &= 1 \\
    y(0) &= 2
\end{cases}
\]

}

\end{minipage}%

\subcaption{First-order system of two exponential decays by composition of
the \texttt{Decay} system. The subsystem \texttt{linked} has a reference
to the outer variable \texttt{x}, while the subsystem
\texttt{independent} defines a new variable with initial condition $2$,
which on the corresponding mathematical expression was named \(y\).}

}

\end{minipage}%

\caption{\label{fig-poincare}Code and corresponding mathematical
expressions for different systems.}

\end{figure}
